\documentclass{article}  
\usepackage{graphicx}
\usepackage{enumerate}
\usepackage{url}
\usepackage[colorlinks,linkcolor=blue]{hyperref}
\usepackage{amsmath}  
\begin{document}
\title{Radio Transients Software Install Instruction}
\author{Chenhui Niu}
\maketitle
\section*{Software in this Document}
After go through a lot of trials and errors, I find a way that could install the following 2 common Pulsar software both on \textbf{Linux OS} and \textbf{MacOS}.  
\begin{itemize}
\item[-] SIGPROC/SIGPYPROC
\item[-] PRESTO
\end{itemize}
Exclude the official instructions on main page of software, I was following two useful memo:
\begin{itemize}
\item[*] \url{https://docs.google.com/document/d/1v8Dm4f-SOeDQX5Yli6syek1pxtqgpw81b1cxqoqv2aU/edit#}

\item[*] \url{http://www.ljtwebdevelopment.com/pulsarref/pulsar-software-install-mac-lion.html}
\end{itemize}

Which are telling us how to install PRESTO on Unbuntu and Pulsar software Suite on Mac separately.
In this Memo,  I quote instructions from above 2 websites in order to make a summary. 

For Fast Radio Transient (FRB) search software, Vishal G. has a Sigproc Giant Pulse Search code which is modified from Sigproc source code. But Here I mainly introduce Heimdall which are running on GPU server. 

\section*{Software Install On MacOS}
\section{Preparation for MacOS}
This is got from:\\

 \url{http://www.ljtwebdevelopment.com/pulsarref/pulsar-software-install-mac-lion.html}

\subsection{Xcode and Homebrew}
Install the latest version of XCode and XCode command line tools from the App store. We use $brew$ to install some software from repository. Follow the install instructions on \url{http://mxcl.github.io/homebrew/}. Homebrew is installed to /usr/local by default.  
\subsection{Install dependencies by brew}
\begin{itemize}
\item[$\$$] brew doctor
\item[$\$$] brew install gfortran
\item[$\$$] brew install glib
\item[$\$$] brew install cfitsio
\item[$\$$] brew install pgplot
\item[$\$$] brew install autoconf
\item[$\$$] brew install automake
\item[$\$$] brew install libtool
\item[$\$$] brew install swig
\item[$\$$] brew install cvs
\item[$\$$] brew install python
\item[$\$$] pip install numpy
\end{itemize}





\subsection{Define some environmental variables}
Add the following lines to your .bash$\_$profile file (if it doesn't exist, create it): \\
****************************************************************************\\

$\#$Path to the pulsar software installation directory eg: 

export ASTROSOFT=/Users/{user}/pulsar$\_$software \\

$\#$ PSRCAT

export PSRCAT$\_$RUNDIR=$\$$ASTROSOFT/psrcat$\_$tar

export PSRCAT$\_$FILE=$\$$ASTROSOFT/psrcat$\_$tar/psrcat.db \\

$\#$ Tempo

export TEMPO=$\$$ASTROSOFT/tempo \\

$\#$ Tempo2

export TEMPO2=$\$$ASTROSOFT/tempo2/T2runtime \\

$\#$ PGPLOT

export PGPLOT$\_$DIR=/usr/local/Cellar/pgplot

export PGPLOT$\_$DEV=/xwindow \\

$\#$ PRESTO

export PRESTO=$\$$ASTROSOFT/presto\\

$\#$ DYLD$\_$LIBRARY$\_$PATH

export DYLD$\_$LIBRARY $\_$ PATH~=~$\$$DYLD$\_$LIBRARY$\_$PATH:$\$$ASTROSOFT/lib:$\$$PRESTO/lib \\

$\#$ PATH 

$\#$ Some Presto executables match sigproc executables so keep separate -

$\#$ all other executables are found in $\$$ASTROSOFT/bin

export PATH=$\$$PATH:/usr/local/git/bin:$\$$ASTROSOFT/bin:$\$$PRESTO/bin
****************************************************************************\\

Then reload your .bash$\_$profile file and check changes are taken up:
\begin{itemize}
\item source $\sim$/.bash$\_$profile
\item echo $\$$ASTROSOFT
\end{itemize}
\section{Sigproc/Sigpyproc Install on MacOs} 
Main page of Sigproc:   
  
 \url{http://sigproc.sourceforge.net/} \\

\subsection{ Dependencies }
\begin{enumerate}[a.]
\item Tempo \\
	Main Page: \url{http://tempo.sourceforge.net/} \\
	Install:
	\begin{enumerate}[1):]
	\item git clone \url{http://git.code.sf.net/p/tempo/tempo} 
	\item cd tempo
	\item ./prepare
	\item ./configure F77=gfortran -~-prefix=$\$$ASTROSOFT
	\item make
	\item make install
	\end{enumerate}		 

\item (Optiion 1 Mainly Suggested)Tempo2\\
	
	Main Page: \url{http://www.atnf.csiro.au/research/pulsar/tempo2/index.php?n=Main.Download} \\
	Install:
	\begin{enumerate}[1):]
	\item git clone \url{https://bitbucket.org/psrsoft/tempo2.git}
	\item cd tempo2
	\item ./bootstrap
	\item ./configure F77=gfortran -~-prefix=$\$$ASTROSOFT
	\item make $\&\&$ make install
	\item make plugins $\&\&$ make plugins-install
	\end{enumerate}	

   (Option 2)Tempo2\\
	
	Main Page: \url{http://tempo.sourceforge.net/} \\
	Install:
	\begin{enumerate}[1):]
	\item wget \url{https://bitbucket.org/psrsoft/tempo2/downloads/tempo2-2017.03.1.tar.gz}
	\item tar -xf tempo2-2017.03.1.tar.gz
	\item cd tempo2
	\item ./bootstrap
	\item ./configure F77=gfortran -~-prefix=$\$$ASTROSOFT
	\item make $\&\&$ make install
	\item make plugins $\&\&$ make plugins-install
	\end{enumerate}	
\item FFTW3\\
	Main Page: \url{http://fftw.org/}
	Install:
	\begin{enumerate}[1):]
	\item wget \url{http://fftw.org/fftw-3.3.7.tar.gz}
	\item tar -xf fftw-3.3.7.tar.gz
	\item cd fftw-3.3.7
	\item ./configure  -~-prefix=$\$$ASTROSOFT
	\item make
	\item make install
	\end{enumerate}		
\item Cfitsio \\
	Already installed from $brew$. Note to specify the lib PATH of Cfitsio:\\
	/usr/local/Cellar/cfitsio/3.420/lib
\item Pgplot
	Already installed from $brew$. Note to specify the lib PATH of PGPlot:\\
	/usr/local/Cellar/pgplot
\end{enumerate}

\subsection{Trick to install Sigproc on Mac}
\subsubsection{Source from Michael Keith's release}
\begin{enumerate}[a):]
\item git clone \url{https://github.com/SixByNine/sigproc.git}
\item ./bootstrap
\item (For MacOs)./configure --prefix=/Users/nch/pulsar$\_$software --with-cfitsio-dir=/usr/local/Cellar/cfitsio/3.420 --with-fftw-dir=$\$$ASTROSOFT F77=gfortran FC=gfortran \\

 (For Centos)./configure --prefix=/Users/nch/pulsar$\_$software --with-cfitsio-dir=/usr/local/Cellar/cfitsio/3.420 --with-fftw-dir=$\$$ASTROSOFT 
\item make
\item make install
\end{enumerate}
\subsubsection{Other Choice}
\begin{enumerate}[1):]
\item  Download source packet from main page:\\
 \url{http://prdownloads.sourceforge.net/sigproc/sigproc-4.3.tar.gz?download}
 \item tar -xf sigproc-4.3.tar.gz
 \item cd sigproc-4.3
 \item ./configure \\
 Enter eg. at prompt to set the default path of the executables.
 \item Edit makefile.darwin with the following:\\
 
   ***********************
 	\begin{enumerate}[a)]
 	\item 	Ensure the PGPLOT libraries in the LPGPLOT line are in the following order, and add -lpng to enable png output from PGPLOT: \\
 	-lcpgplot -lpgplot -lpng
 	\item  Add the following line defining the fortran compiler: \\
 	FC = gfortran -ffixed-line-length-none
 	\item Uncomment LFITS and LFFTW and edit paths to: \\
 	
 	LFITS = -L/path/to/lib/ -lcfitsio  \\ 
 	LFFTW = -L/path/to/lib/ -lfftw3 -lfftw3f \\
 	
 	Here Our path are:\\
 	
 	LFITS = -L/usr/local/Cellar/cfitsio/3.420/lib -lcfitsio\\ 
    LFFTW = -L$\$$(ASTROSOFT)/lib -lfftw3 -lfftw3f

 	\end{enumerate}
 	***********************\\    
\item Remove the backslash and quote from dosearch.f (line 265):\\
	Change from:\\
	write(llog,*) 'DB\'~s slow-but-simple harmonic summing routine' \\
	to:\\
	write(llog,*) 'DBs slow-but-simple harmonic summing routine'
\item Edit makefile:\\
***********************\\
	include makefile.$\$$(OSTYPE) \\

	to:\\

	include makefile.darwin\\
	
	LIB = libsigproc$\_\$$(OSTYPE).a \\

	to:\\

	LIB = libsigproc$\_$darwin.a \\
***********************\\	
\item Save the file and compile it:
   \begin{enumerate}
 	 	\item[$\$$] make
 	 	\item[$\$$] make quickplot
   \end{enumerate}

\end{enumerate}

\subsection{Use Sigproc and Sigpyproc}
Following the steps above , sigproc should be work.  For Sigpyproc which are a python version for sigproc, I have written a memo in the early time. you can find it from :\\
\url{https://github.com/peterniuzai/Work_memo/raw/master/SIGPYPROC_MEMO.pdf}\\
It has description for both Mac and Linux server.
The manual of Sigproc could found from:\\
\url{http://sigproc.sourceforge.net/sigproc.pdf}\\
The use of Sigpyproc could found from:\\
\url{http://www3.mpifr-bonn.mpg.de/staff/ebarr/sigpyproc/Introduction.html}

\section{Presto install on Mac}
Main Page:\\
\url{http://www.cv.nrao.edu/~sransom/presto/}\\
Get the latest version of Presto from the main page:
\begin{enumerate}[1)]
\item git clone git://github.com/scottransom/presto.git
\item cd presto
\item git pull 
\end{enumerate}
\subsection{Dependencies}
Presto has the dependencies same as Sigproc. 
\subsection{Trick to install Presto on Mac}
\begin{enumerate}[1):]
\item cd presto/src
\item Edit Makefile as follows: \\
	\begin{enumerate}[a)]
	\item Define paths to fftw and cfitsio includes and libs:\\
		\begin{itemize}
		\item[-] FFTINC = -I$\$$(ASTROSOFT)/include
		\item[-] FFTLINK = -L$\$$(ASTROSOFT)/lib -lfftw3f
		\item[-] CFITSIOINC = -I/usr/local/Cellar/cfitsio/3.420/include
		\item[-] CFITSIOLINK = -L/usr/local/Cellar/cfitsio/3.420/lib/ -lcfitsio
		\end{itemize}
	\item Add -lm flag to CFLAGS: \\
		CFLAGS = -I$\$$(PRESTO)/include~$\$$(GLIBINC)~$\$$(CFITSIOINC) $\$$(PGPLOTINC) $\$$(FFTINC) \
        -DUSEFFTW -DUSEMMAP -D$\_$LARGEFILE$\_$SOURCE -D$\_$FILE$\_$OFFSET$\_$BITS=64 \
        -g -Wall -W -fPIC -O3 -ffast-math -Wall -W -fPIC -lm
	\end{enumerate}
\item Continue the build:
  \begin{itemize}
    \item[$\$$] make makewisdom
    \item[$\$$] make prep
    \item[$\$$] make 
   \end{itemize}
\end{enumerate}
\subsection{Use Presto}
There's a nice PPT to show us how to use Presto:
\url{http://www.cv.nrao.edu/~sransom/PRESTO_search_tutorial.pdf}

\section*{Install Pulsar software on Linux server Ubuntu/Centos}
\section{Install Presto on Ubuntu}
Refer from:\\
 \url{https://docs.google.com/document/d/1v8Dm4f-SOeDQX5Yli6syek1pxtqgpw81b1cxqoqv2aU/edit#} \\
 
\subsection{Get dependencies ready}
sudo apt-get install git libfftw3-bin libfftw3-dbg libfftw3-dev libfftw3-doc libfftw3-double3 libfftw3-long3 libfftw3-quad3 libfftw3-single3 pgplot5 csh automake gfortran libglib2.0-dev libccfits-dev libcfitsio3 libcfitsio3-dev libx11-dev libpng12-dev nvidia-cuda-dev libcuda1-331 -y
\subsection{Install steps}
\begin{enumerate}[Step 1)]
\item git clone git://github.com/scottransom/presto.git
\item git clone git://git.code.sf.net/p/tempo/tempo
\item cd tempo
\item ./prepare
\item ./configure --prefix=/usr/local/
\item make $\&\&$ sudo make install
\item cp tempo.cfg src/
\item cp tempo.hlp src/
\item sudo vi /etc/environment
		\begin{enumerate}[*]
		\item TEMPO=”/home/nch/work/tempo/src”
		\item PRESTO=”/home/nch/work/presto”
		\it	em PGPLOT$\_$DIR=”/usr/lib/pgplot5”
		\item FFTW$\_$PATH=”/usr/”
		\item LD$\_$LIBRARY$\_$PATH=”/usr/local/cuda/lib64:/usr/local/cuda/lib:/home/nch/work/presto/lib”
		\item PYTHONPATH=”/home/nch/work/presto/python:/home/nch/work/presto/lib/python”
		\end{enumerate}
\item sudo reboot
\item cd ~/Downloads/presto/src
\item make makewisdom
\item make prep
\item make 
\end{enumerate}
\subsection{Test Presto}
\begin{itemize}
\item[$\$$] wget  \url{http://www.cv.nrao.edu/~sransom/GBT_Lband_PSR.fil} \\
\item[$\$$] readfile GBT$\_$Lband$\_$PSR.fil \\
\item[$\$$] rfifind -time 2.0 -o Lband GBT$\_$Lband$\_$PSR.fil \\
\item[$\$$] prepfold -n 64 -nsub 96 -p 0.004621638 -dm 62.0 GBT$\_$Lband$\_$PSR.fil
\end{itemize}

\section{Install Presto on Centos}
This is similar as Ubuntu. Different thing is some dependency software yum doesn't have.\\

1)you need install Cfitsio from \url{https://heasarc.gsfc.nasa.gov/fitsio/} \\
Here I use the latest version: cfitsio$\_$latest.tar.gz \\

2) FFTW, you could find this from MacOS instruction. \\
( ./configure --prefix=/usr/local/ --enable-float --enable-threads --enable-shared CFLAGS=-fPIC FFLAGS=-fPIC) \\

3) Install libpng-code: \\
	url{http://libpng.org/pub/png/libpng.html}\\
    ./configure [--prefix=/path] \\

    make check \\

    make install

4) PGplot  \\

	This is a tricky thing. refer to:\\
	 \url{https://www.cnblogs.com/shaoguangleo/archive/2012/04/09/2806095.html} \\
	 \begin{enumerate}[1)]
	 \item sudo yum install libX11-devel
	 \item sudo yum install gcc-gfortran 
	 \item cd /usr/local/src
	 \item mv ~/Downloads/pgplot5.2.tar.gz .
	 \item tar zxvf pgplot5.2.tar.gz
	 \item mkdir /usr/local/pgplot
	 \item cd /usr/local/pgplot
	 \item cp /usr/local/src/pgplot/drivers.list . 
	 \item Edit drives.list , Remove the "!"  before :\\
	 /PS, /VPS, /CPS, /VCPS and /XServe /XWINDOW 
	 \item Do the following command in dir: /usr/local/pgplot\\
	 /usr/local/src/pgplot/makemake /usr/local/src/pgplot linux g77$\_$gcc$\_$aout
	 \item Edit makefile :
	 \item Instead of FCOMPL=g77 by FCOMPL=gfortran
	 \item make $\&\&$ make cpg 
	 \item Set the environment: \\
	 	\begin{itemize}
	 	\item[$\$$] export PGPLOT$\_$DIR=/usr/local/pgplot
	 	\item[$\$$] export PGPLOT$\_$DEV=/Xserve
	 	\end{itemize}
	 \item Test:\\	
	 	\begin{itemize}
	 	\item[$\$$] /usr/local/pgplot/pgdemo1
	 	\item[$\$$] /usr/local/pgplot/pgdemo2
	 	\item[$\$$] ...
	 	\end{itemize}
	 \end{enumerate}
After install the dependencies above, then set the environment. Just follow the INSTALL file in PRESTO source. Remember use '--prefix=/usr/local/' when configure.

\section{Common Problems}
\begin{enumerate}[(1)]
\item "Can't find presto.so" : \\
Check if LD$\_$LiBRARY$\_$PATH variable defined.if not ,defined it in .bashrc \\
export LD$\_$LIBRARY$\_$PATH="$\$$LD$\_$LIBRARY$\_$PATH:$\$$HOME/work/sigpyproc/lib/c"
\item  "undefined reference to curl$|\_$global$\_$init, curl$\_$easy$\_$init and other function(C)" \\
\url{https://stackoverflow.com/questions/16476196/undefined-reference-to-curl-global-init-curl-easy-init-and-other-functionc} \\
gcc -lcurl test.c 
\item Open un-recognized observatory: \\
Add observatory info in /tempo/obsys.dat  \\
In $\$$presto/src/ , open misc$\_$utils.c file.  \\
modify function: telescope$\_$to$\_$tempocode() , add the observatory info.\\
Recompile presto again.

\end{enumerate}

\end{document}